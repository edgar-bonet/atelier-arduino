\documentclass{article}
\usepackage[utf8]{inputenc}
\usepackage[T1]{fontenc}
\usepackage{minted}
\usemintedstyle{algol}
\addtolength{\oddsidemargin}{-10mm}
\addtolength{\textwidth}{20mm}
\addtolength{\topmargin}{-10mm}
\addtolength{\textheight}{20mm}

\begin{document}

\centerline{\LARGE Pense-bête Arduino/C++}
\vspace{10mm}

\section*{Programme (« sketch »)}

\begin{minted}{cpp}
#include <bibliothèque>
...

déclarations de variables et fonctions

// Code exécuté au démarrage du programme.
void setup() {
    instructions...
}

// Code exécuté à répétition, sans relâche.
void loop() {
    instructions...
}
\end{minted}

\section*{Déclaration de variable}

\begin{minted}{cpp}
type nom = valeur_initiale;

const type nom = valeur;

enum { nom_valeur, nom_valeur, ... } nom_variable = valeur_initiale;
\end{minted}
où
\texttt{type} = \textbf{int}, \textbf{unsigned int}, \textbf{long},
\textbf{unsigned long}, \textbf{float}, ...

\section*{Déclaration de fonction}

\begin{minted}{cpp}

// Fonction qui ne renvoie pas de valeur.
void nom_de_fonction(paramètre, ...)
{
    déclarations de variables locales...
    instructions...
}

// Fonction qui renvoie une valeur.
type_de_retour nom_de_fonction(paramètre, ...)
{
    déclarations de variables locales...
    instructions...
    return expression;
}
\end{minted}

\section*{Instructions}

\begin{minted}{cpp}
variable = expression;  // affectation

fonction(argument, ...);  // appel de fonction

if (condition) {
    instructions...  // code exécuté de façon conditionnelle
}

if (condition) {
    instructions...  // exécuté si la condition est vraie
} else {
    instructions...  // exécuté si la condition est fausse
}

while (condition) {
    instructions...  // code exécuté à répétition
}

switch (expression) {  // choix d'un cas parmi un ensemble
    case valeur:
        instructions...
        break;
    case valeur:
        instructions...
        break;
    ...
}

{  // regroupement de plusieurs instructions
    instructions...
}
\end{minted}

\end{document}
